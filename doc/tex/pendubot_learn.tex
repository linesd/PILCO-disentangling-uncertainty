
% This LaTeX was auto-generated from an M-file by MATLAB.
% To make changes, update the M-file and republish this document.



    
    
      \subsection{pendubot\_learn.m}

\begin{par}
\textbf{Summary:} Script to learn a controller for the pendubot swingup (a double pendulum with the inner joint actuated)
\end{par} \vspace{1em}
\begin{par}
Copyright (C) 2008-2013 by Marc Deisenroth, Andrew McHutchon, Joe Hall, and Carl Edward Rasmussen.
\end{par} \vspace{1em}
\begin{par}
Last modified: 2013-03-27
\end{par} \vspace{1em}


\subsection*{High-Level Steps} 

\begin{enumerate}
\setlength{\itemsep}{-1ex}
   \item Load parameters
   \item Create J initial trajectories by applying random controls
   \item Controlled learning (train dynamics model, policy learning, policy application)
\end{enumerate}


\subsection*{Code} 


\begin{lstlisting}
% 1. Initialization
clear all; close all;
settings_pendubot;            % load scenario-specific settings
basename = 'pendubot_';       % filename used for saving data

% 2. Initial J random rollouts
for jj = 1:J
  [xx, yy, realCost{jj}, latent{jj}] = ...
    rollout(gaussian(mu0, S0), struct('maxU',policy.maxU), H, plant, cost);
  x = [x; xx]; y = [y; yy];       % augment training sets for dynamics model
  if plotting.verbosity > 0;      % visualization of trajectory
    if ~ishandle(1); figure(1); else set(0,'CurrentFigure',1); end; clf(1);
    draw_rollout_pendubot;
  end
end

mu0Sim(odei,:) = mu0; S0Sim(odei,odei) = S0;
mu0Sim = mu0Sim(dyno); S0Sim = S0Sim(dyno,dyno);

% 3. Controlled learning (N iterations)
for j = 1:N
  trainDynModel;   % train (GP) dynamics model
  learnPolicy;     % learn policy
  applyController; % apply controller to system
  disp(['controlled trial # ' num2str(j)]);
  if plotting.verbosity > 0;      % visualization of trajectory
    if ~ishandle(1); figure(1); else set(0,'CurrentFigure',1); end; clf(1);
    draw_rollout_pendubot;
  end
end
\end{lstlisting}
