
% This LaTeX was auto-generated from an M-file by MATLAB.
% To make changes, update the M-file and republish this document.



    
    

\subsection*{covSum.m} 

\begin{par}
\textbf{Summary:} Compose a covariance function as the sum of other covariance functions. This function doesn't actually compute very much on its own, it merely does some bookkeeping, and calls other covariance functions to do the actual work.
\end{par} \vspace{1em}

\begin{verbatim}   function [A, B] = covSum(covfunc, logtheta, x, z)\end{verbatim}
    \begin{par}
(C) Copyright 2006 by Carl Edward Rasmussen, 2006-03-20.
\end{par} \vspace{1em}

\begin{lstlisting}
function [A, B] = covSum(covfunc, logtheta, x, z)
\end{lstlisting}


\subsection*{Code} 


\begin{lstlisting}
for i = 1:length(covfunc)                   % iterate over covariance functions
  f = covfunc(i);
  if iscell(f{:}), f = f{:}; end          % dereference cell array if necessary
  j(i) = cellstr(feval(f{:}));
end

if nargin == 1,                                   % report number of parameters
  A = char(j(1)); for i=2:length(covfunc), A = [A, '+', char(j(i))]; end
  return
end

[n, D] = size(x);

v = [];              % v vector indicates to which covariance parameters belong
for i = 1:length(covfunc), v = [v repmat(i, 1, eval(char(j(i))))]; end

switch nargin
case 3                                              % compute covariance matrix
  A = zeros(n, n);                       % allocate space for covariance matrix
  for i = 1:length(covfunc)                  % iteration over summand functions
    f = covfunc(i);
    if iscell(f{:}), f = f{:}; end        % dereference cell array if necessary
    A = A + feval(f{:}, logtheta(v==i), x);            % accumulate covariances
  end

case 4                      % compute derivative matrix or test set covariances
  if nargout == 2                                % compute test set cavariances
    A = zeros(size(z,1),1); B = zeros(size(x,1),size(z,1));    % allocate space
    for i = 1:length(covfunc)
      f = covfunc(i);
      if iscell(f{:}), f = f{:}; end      % dereference cell array if necessary
      [AA BB] = feval(f{:}, logtheta(v==i), x, z);   % compute test covariances
      A = A + AA; B = B + BB;                                  % and accumulate
    end
  else                                            % compute derivative matrices
    i = v(z);                                       % which covariance function
    j = sum(v(1:z)==i);                    % which parameter in that covariance
    f = covfunc(i);
    if iscell(f{:}), f = f{:}; end        % dereference cell array if necessary
    A = feval(f{:}, logtheta(v==i), x, j);                 % compute derivative
  end

end
\end{lstlisting}
