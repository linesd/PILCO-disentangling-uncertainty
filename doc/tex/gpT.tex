
% This LaTeX was auto-generated from an M-file by MATLAB.
% To make changes, update the M-file and republish this document.



    
    

\subsection*{gpT.m} 

\begin{par}
\textbf{Summary:} Test derivatives of gp*-family of functions. It is assumed that the gp* function computes the mean and the variance of a GP prediction for a Gaussian distributed input $x\sim\mathcal N(m,s)$. The GP-family of functions is located in \begin{verbatim}rootDir\end{verbatim}/gp and is called gp*.m
\end{par} \vspace{1em}
\begin{verbatim}function [dd dy dh] = gpT(deriv, gp, m, s, delta)\end{verbatim}
\begin{par}
\textbf{Input arguments:}
\end{par} \vspace{1em}
\begin{verbatim}deriv    desired derivative. options:
     (i)    'dMdm' - derivative of the mean of the GP prediction
             wrt the mean of the input distribution
     (ii)   'dMds' - derivative of the mean of the GP prediction
             wrt the variance of the input distribution
     (iii)  'dMdp' - derivative of the mean of the GP prediction
             wrt the GP parameters
     (iv)   'dSdm' - derivative of the variance of the GP prediction
             wrt the mean of the input distribution
     (v)    'dSds' - derivative of the variance of the GP prediction
             wrt the variance of the input distribution
     (vi)   'dSdp' - derivative of the variance of the GP prediction
             wrt the GP parameters
     (vii)  'dVdm' - derivative of inv(s)*(covariance of the input and the
             GP prediction) wrt the mean of the input distribution
     (viii) 'dVds' - derivative of inv(s)*(covariance of the input and the
             GP prediction) wrt the variance of the input distribution
     (ix)   'dVdp' - derivative of inv(s)*(covariance of the input and the
             GP prediction) wrt the GP parameters
gp       GP structure
  .fcn   function handle to the GP function used for predictions at
         uncertain inputs
  .\ensuremath{<}\ensuremath{>}    other fields that are passed on to the GP function
m        mean of the input distribution
s        covariance of the input distribution
delta    (optional) finite difference parameter. Default: 1e-4\end{verbatim}
\begin{par}
\textbf{Output arguments:}
\end{par} \vspace{1em}
\begin{verbatim}dd         relative error of analytical vs. finite difference gradient
dy         analytical gradient
dh         finite difference gradient\end{verbatim}
\begin{par}
Copyright (C) 2008-2013 by Marc Deisenroth, Andrew McHutchon, Joe Hall, and Carl Edward Rasmussen.
\end{par} \vspace{1em}
\begin{par}
Last modified: 2013-06-07
\end{par} \vspace{1em}

\begin{lstlisting}
function [dd dy dh] = gpT(deriv, gp, m, s, delta)
\end{lstlisting}


\subsection*{Code} 


\begin{lstlisting}
% set up a default training set and input distribution if not passed in
if nargin < 2
  nn = 1000; np = 100;
  D = 5; E = 4;         % input and predictive dimensions

  gp.fcn = @gp0d;
  gp.hyp = [randn(D+1,E); zeros(1,E)];
  gp.inputs = randn(nn,D);
  gp.targets = randn(nn,E);
  gp.induce = randn(np,D,E);
end

if nargin < 3           % no input distribution specified
  if isfield(gp, 'p')   % if gp is a policy, extract targets/inputs
    gp.inputs = gp.p.inputs; gp.targets = gp.p.targets;
  end
  D = size(gp.inputs, 2);
  m = randn(D,1); s = randn(D); s = s*s';
end

if nargin < 5; delta = 1e-4; end
D = length(m);                                                      % input size


% check derivatives
switch deriv

  case 'dMdm'
    [dd dy dh] = checkgrad(@gpT0, m, delta, gp, s);

  case 'dSdm'
    [dd dy dh] = checkgrad(@gpT1, m, delta, gp, s);

  case 'dVdm'
    [dd dy dh] = checkgrad(@gpT2, m, delta, gp, s);

  case 'dMds'
    [dd dy dh] = checkgrad(@gpT3, s(tril(ones(D))==1), delta, gp, m);

  case 'dSds'
    [dd dy dh] = checkgrad(@gpT4, s(tril(ones(D))==1), delta, gp, m);

  case 'dVds'
    [dd dy dh] = checkgrad(@gpT5, s(tril(ones(D))==1), delta, gp, m);

  case 'dMdp'
    p = unwrap(gp);
    [dd dy dh] = checkgrad(@gpT6, p, delta, gp, m, s) ;

  case 'dSdp'
    p = unwrap(gp);
    [dd dy dh] = checkgrad(@gpT7, p, delta, gp, m, s) ;

  case 'dVdp'
    p = unwrap(gp);
    [dd dy dh] = checkgrad(@gpT8, p, delta, gp, m, s) ;

end
\end{lstlisting}

\begin{lstlisting}
function [f, df] = gpT0(m, gp, s)                             % dMdm
if nargout == 1
  M = gp.fcn(gp, m, s);
else
  [M, S, V, dMdm] = gp.fcn(gp, m, s);
  df = dMdm;
end
f = M;

function [f, df] = gpT1(m, gp, s)                             % dSdm
if nargout == 1
  [M, S] = gp.fcn(gp, m, s);
else
  [M, S, V, dMdm, dSdm] = gp.fcn(gp, m, s);
  df = dSdm;
end
f = S;

function [f, df] = gpT2(m, gp, s)                             % dVdm
if nargout == 1
  [M, S, V] = gp.fcn(gp, m, s);
else
  [M, S, V, dMdm, dSdm, dVdm] = gp.fcn(gp, m, s);
  df = dVdm;
end
f = V;

function [f, df] = gpT3(s, gp, m)                             % dMds
d = length(m);
v(tril(ones(d))==1) = s; s = reshape(v,d,d); s = s+s'-diag(diag(s));
if nargout == 1
  M = gp.fcn(gp, m, s);
else
  [M, S, V, dMdm, dSdm, dVdm, dMds] = gp.fcn(gp, m, s);
  dd = length(M); dMds = reshape(dMds,dd,d,d); df = zeros(dd,d*(d+1)/2);
  for i=1:dd;
    dMdsi(:,:) = dMds(i,:,:); dMdsi = dMdsi + dMdsi'-diag(diag(dMdsi));
    df(i,:) = dMdsi(tril(ones(d))==1);
  end
end
f = M;

function [f, df] = gpT4(s, gp, m)                             % dSds
d = length(m);
v(tril(ones(d))==1) = s; s = reshape(v,d,d); s = s+s'-diag(diag(s));
if nargout == 1
  [M, S] = gp.fcn(gp, m, s);
else
  [M, S, C, dMdm, dSdm, dCdm, dMds, dSds] = gp.fcn(gp, m, s);
  dd = length(M); dSds = reshape(dSds,dd,dd,d,d); df = zeros(dd,dd,d*(d+1)/2);
  for i=1:dd; for j=1:dd
      dSdsi(:,:) = dSds(i,j,:,:); dSdsi = dSdsi+dSdsi'-diag(diag(dSdsi));
      df(i,j,:) = dSdsi(tril(ones(d))==1);
    end; end
end
f = S;

function [f, df] = gpT5(s, gp, m)                             % dVds
d = length(m);
v(tril(ones(d))==1) = s; s = reshape(v,d,d); s = s+s'-diag(diag(s));
if nargout == 1
  [M, S, V] = gp.fcn(gp, m, s);
else
  [M, S, V, dMdm, dSdm, dVdm, dMds, dSds, dVds] = gp.fcn(gp, m, s);
  dd = length(M); dVds = reshape(dVds,d,dd,d,d); df = zeros(d,dd,d*(d+1)/2);
  for i=1:d; for j=1:dd
      dCdsi = squeeze(dVds(i,j,:,:)); dCdsi = dCdsi+dCdsi'-diag(diag(dCdsi));
      df(i,j,:) = dCdsi(tril(ones(d))==1);
    end; end
end
f = V;

function [f, df] = gpT6(p, gp, m, s)                          % dMdp
gp = rewrap(gp, p);
if nargout == 1
  M = gp.fcn(gp, m, s);
else
  [M, S, V, dMdm, dSdm, dVdm, dMds, dSds, dVds, dMdp] = ...
    gp.fcn(gp, m, s);
  df = dMdp;
end
f = M;

function [f, df] = gpT7(p, gp, m, s)                          % dSdp
gp = rewrap(gp, p);
if nargout == 1
  [M, S] = gp.fcn(gp, m, s);
else
  [M, S, V, dMdm, dSdm, dVdm, dMds, dSds, dVds, dMdp, dSdp] = ...
    gp.fcn(gp, m, s);
  df = dSdp;
end
f = S;

function [f, df] = gpT8(p, gp, m, s)
gp = rewrap(gp, p);
if nargout == 1
  [M, S, V] = gp.fcn(gp, m, s);
else
  [M, S, V, dMdm, dSdm, dVdm, dMds, dSds, dVds, dMdp, dSdp, dVdp] = ...
    gp.fcn(gp, m, s);
  df = dVdp;

end
f = V;
\end{lstlisting}
