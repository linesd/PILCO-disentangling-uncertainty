
% This LaTeX was auto-generated from an M-file by MATLAB.
% To make changes, update the M-file and republish this document.



    
    
      \subsection{rollout.m}

\begin{par}
\textbf{Summary:} Generate a state trajectory using an ODE solver (and any additional dynamics) from a particular initial state by applying either a particular policy or random actions.
\end{par} \vspace{1em}
\begin{verbatim}function [x y L latent] = rollout(start, policy, H, plant, cost)\end{verbatim}
\begin{par}
\textbf{Input arguments:}
\end{par} \vspace{1em}
\begin{verbatim}start       vector containing initial states (without controls)   [nX  x  1]
policy      policy structure
  .fcn        policy function
  .p          parameter structure (if empty: use random actions)
  .maxU       vector of control input saturation values           [nU  x  1]
H           rollout horizon in steps
plant       the dynamical system structure
  .subplant   (opt) additional discrete-time dynamics
  .augment    (opt) augment state using a known mapping
  .constraint (opt) stop rollout if violated
  .poli       indices for states passed to the policy
  .dyno       indices for states passed to cost
  .odei       indices for states passed to the ode solver
  .subi       (opt) indices for states passed to subplant function
  .augi       (opt) indices for states passed to augment function
cost    cost structure\end{verbatim}
\begin{par}
\textbf{Output arguments:}
\end{par} \vspace{1em}
\begin{verbatim}x          matrix of observed states                           [H   x nX+nU]
y          matrix of corresponding observed successor states   [H   x   nX ]
L          cost incurred at each time step                     [ 1  x    H ]
latent     matrix of latent states                             [H+1 x   nX ]\end{verbatim}
\begin{par}
Copyright (C) 2008-2013 by Marc Deisenroth, Andrew McHutchon, Joe Hall, and Carl Edward Rasmussen.
\end{par} \vspace{1em}
\begin{par}
Last modification: 2013-05-21
\end{par} \vspace{1em}


\subsection*{High-Level Steps} 

\begin{enumerate}
\setlength{\itemsep}{-1ex}
   \item Compute control signal $u$ from state $x$: either apply policy or random actions
   \item Simulate the true dynamics for one time step using the current pair $(x,u)$
   \item Check whether any constraints are violated (stop if true)
   \item Apply random noise to the successor state
   \item Compute cost (optional)
   \item Repeat until end of horizon
\end{enumerate}

\begin{lstlisting}
function [x y L latent] = rollout(start, policy, H, plant, cost)
\end{lstlisting}


\subsection*{Code} 


\begin{lstlisting}
if isfield(plant,'augment'), augi = plant.augi;             % sort out indices!
else plant.augment = inline('[]'); augi = []; end
if isfield(plant,'subplant'), subi = plant.subi;
else plant.subplant = inline('[]',1); subi = []; end
odei = plant.odei; poli = plant.poli; dyno = plant.dyno; angi = plant.angi;
simi = sort([odei subi]);
nX = length(simi)+length(augi); nU = length(policy.maxU); nA = length(angi);

state(simi) = start; state(augi) = plant.augment(state);      % initializations
x = zeros(H+1, nX+2*nA);
x(1,simi) = start' + randn(size(simi))*chol(plant.noise);
x(1,augi) = plant.augment(x(1,:));
u = zeros(H, nU); latent = zeros(H+1, size(state,2)+nU);
y = zeros(H, nX); L = zeros(1, H); next = zeros(1,length(simi));

for i = 1:H % --------------------------------------------- generate trajectory
  s = x(i,dyno)'; sa = gTrig(s, zeros(length(s)), angi); s = [s; sa];
  x(i,end-2*nA+1:end) = s(end-2*nA+1:end);

  % 1. Apply policy ... or random actions --------------------------------------
  if isfield(policy, 'fcn')
    u(i,:) = policy.fcn(policy,s(poli),zeros(length(poli)));
  else
    u(i,:) = policy.maxU.*(2*rand(1,nU)-1);
  end
  latent(i,:) = [state u(i,:)];                                  % latent state

  % 2. Simulate dynamics -------------------------------------------------------
  next(odei) = simulate(state(odei), u(i,:), plant);
  next(subi) = plant.subplant(state, u(i,:));

  % 3. Stop rollout if constraints violated ------------------------------------
  if isfield(plant,'constraint') && plant.constraint(next(odei))
    H = i-1;
    fprintf('state constraints violated...\n');
    break;
  end

  % 4. Augment state and randomize ---------------------------------------------
  state(simi) = next(simi); state(augi) = plant.augment(state);
  x(i+1,simi) = state(simi) + randn(size(simi))*chol(plant.noise);
  x(i+1,augi) = plant.augment(x(i+1,:));

  % 5. Compute Cost ------------------------------------------------------------
  if nargout > 2
    L(i) = cost.fcn(cost,state(dyno)',zeros(length(dyno)));
  end
end

y = x(2:H+1,1:nX); x = [x(1:H,:) u(1:H,:)];
latent(H+1, 1:nX) = state; latent = latent(1:H+1,:); L = L(1,1:H);
\end{lstlisting}
